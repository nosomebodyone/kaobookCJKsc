% Load the kaobook class
\documentclass[
	chinese,
	fontsize=10pt, % Base font size
	twoside=true, % Use different layouts for even and odd pages (in particular, if twoside=true, the margin column will be always on the outside)
	open=any, % If twoside=true, uncomment this to force new chapters to start on any page, not only on right (odd) pages
	secnumdepth=1, % How deep to number headings. Defaults to 1 (sections)
]{styles/kaobookCJKsc}

%----------------------------------------------------------------------------------------
%	包以及文档配置,新项目可以直接复制粘贴
%----------------------------------------------------------------------------------------

% 加载排版包(不支持 chinese)
\usepackage{polyglossia}
\setmainlanguage{english}
\SetLanguageKeys{english}{indentfirst=true}
% 加载引号风格包(不支持 chinese)
\usepackage[style=english]{csquotes}
% 加载参考文献相关设置
\usepackage[style=gb7714-2015ms,backref=false]{styles/kaobiblioCJKsc} % 默认使用 GB 标准
\addbibresource{main.bib} % 参考文献文本
% 加载数学相关设置
\usepackage[framed]{styles/kaotheoremsCJKsc}
% 加载超链接相关设置
\usepackage{styles/kaorefsCJKsc}
% 加载平台中文字体及相关设置
\usepackage[testMathBox]{styles/kaoWinCJKsc}
% 加载测试包
\usepackage{blindtext}

% 图片路径
\graphicspath{{examples/documentation/images/}{images/}}
% 生成用于编译 索引 的文件
\makeindex[columns=3, title=索引, intoc]
% 生成用于编译 术语 的文件
\makenomenclature
% 每章开头重新给 边注 编号
\counterwithin*{sidenote}{chapter}
% 允许标题中出现 \\ 而不发出警告
\pdfstringdefDisableCommands{\let\\\empty}

%----------------------------------------------------------------------------------------
\begin{document}

%----------------------------------------------------------------------------------------
%	加载其他无法放在导言区的配置
%----------------------------------------------------------------------------------------
%%%%%%%%%%%%%%%无法放在导言区的设置%%%%%%%%%%%%%%%%%%%%
%目录中文名
\renewcommand{\contentsname}{目录}
\renewcommand{\listfigurename}{插图}
\renewcommand{\listtablename}{表格}
\renewcommand{\lstlistlistingname}{清单}
\renewcommand{\lstlistingname}{清单}
\renewcommand{\indexname}{索引}
\renewcommand{\bibname}{参考文献}
%其他中文名
\renewcommand{\pagename}{页}
\renewcommand{\chaptername}{章}
\renewcommand{\sectionname}{节}
\renewcommand{\subsectionname}{小节}
\renewcommand{\figurename}{图}
\renewcommand{\tablename}{表}
\renewcommand{\eqname}{公式}
\renewcommand{\defname}{定义}
\renewcommand{\assumname}{假设}
\renewcommand{\thmname}{定理}
\renewcommand{\propname}{命题}
\renewcommand{\lemmaname}{引理}
\renewcommand{\remarkname}{备注}
\renewcommand{\examplename}{例}
\renewcommand{\exercisename}{练习}
%引用语序
\renewcommand{\refpage}[1]{\hyperref[#1]{\textcolor{Blue}{第\xspace\pageref{page:#1}\xspace\pagename}}} % 第 84 页
\renewcommand{\refch}[1]{\hyperref[ch:#1]{\textcolor{Blue}{第\xspace\ref{ch:#1}\xspace\chaptername}}} % 第 7 章
\renewcommand{\nrefch}[1]{\hyperref[ch:#1]{\textcolor{Blue}{第\xspace\ref{ch:#1}\xspace\chaptername(\nameref{ch:#1})}}} % 第 7 章(数学)
\renewcommand{\refsec}[1]{\hyperref[sec:#1]{\textcolor{Blue}{第\xspace\ref{sec:#1}\xspace\sectionname}}}
\renewcommand{\nrefsec}[1]{\hyperref[sec:#1]{\textcolor{Blue}{第\xspace\ref{sec:#1}\xspace\sectionname(\nameref{sec:#1})}}}
\renewcommand{\refsubsec}[1]{\hyperref[subsec:#1]{\textcolor{Blue}{第\xspace\ref{subsec:#1}\xspace\subsectionname}}}
\renewcommand{\nrefsubsec}[1]{\hyperref[subsec:#1]{\textcolor{Blue}{第\xspace\ref{subsec:#1}\xspace\subsectionname(\nameref{subsec:#1})}}}
%脚注编号
% \renewcommand{\thefootnote}{\arabic{footnote}} % 1, 2, 3
% \renewcommand{\thefootnote}{\alph{footnote}} % a, b, c
% \renewcommand{\thefootnote}{\roman{footnote}} % i, ii, iii




%----------------------------------------------------------------------------------------
%	BOOK INFORMATION
%----------------------------------------------------------------------------------------

\titlehead{一本小书}
\title[用于制作书籍的 kaobookCJK 模板]{用于制作书籍的 kaobookCJK 模板}
\author[MK]{薛浩}
\date{\today}
\publishers{绝绝子出版社}

%----------------------------------------------------------------------------------------

\frontmatter % Denotes the start of the pre-document content, uses roman numerals

%----------------------------------------------------------------------------------------
%	COPYRIGHT PAGE
%----------------------------------------------------------------------------------------

\makeatletter
\uppertitleback{
	\textbf{内容提要}
	
	\bigskip
	
	介绍一份可以制作精美图书的 \LaTeX 模板
}

\lowertitleback{
	\textbf{声明} \\
	You can edit this page to suit your needs. For instance, here we have a no copyright statement, a colophon and some other information. This page is based on the corresponding page of Ken Arroyo Ohori's thesis, with minimal changes.
	
	\medskip
	
	\textbf{版权没有,翻版不究} \\
	\cczero\ This book is released into the public domain using the CC0 code. To the extent possible under law, I waive all copyright and related or neighbouring rights to this work.
	
	To view a copy of the CC0 code, visit: \\\url{http://creativecommons.org/publicdomain/zero/1.0/}
	
	\medskip
	
	\textbf{版本} \\
	This document was typeset with the help of \href{https://sourceforge.net/projects/koma-script/}{\KOMAScript} and \href{https://www.latex-project.org/}{\LaTeX} using the \href{https://github.com/fmarotta/kaobook/}{kaobook} class.
	
	\medskip
	
	\textbf{出版} \\
	First printed in \today by \@publishers
}
\makeatother

%----------------------------------------------------------------------------------------
%	DEDICATION
%----------------------------------------------------------------------------------------

\dedication{
	世界的和谐体现在形式和数量上,心灵和自然哲学的所有诗意都体现在数学美的概念中。\\
	\flushright -- 达西·温特沃斯·汤普森 
}

%----------------------------------------------------------------------------------------
%	OUTPUT TITLE PAGE AND PREVIOUS
%----------------------------------------------------------------------------------------

% Note that \maketitle outputs the pages before here
\maketitle

%----------------------------------------------------------------------------------------
%	PREFACE
%----------------------------------------------------------------------------------------

\chapter*{前言}

\blindtext

%----------------------------------------------------------------------------------------
%	TABLE OF CONTENTS & LIST OF FIGURES/TABLES
%----------------------------------------------------------------------------------------

\begingroup % Local scope for the following commands

% Define the style for the TOC, LOF, and LOT
%\setstretch{1} % Uncomment to modify line spacing in the ToC
%\hypersetup{linkcolor=blue} % Uncomment to set the colour of links in the ToC
\setlength{\textheight}{230\vscale} % Manually adjust the height of the ToC pages

% Turn on compatibility mode for the etoc package
\etocstandarddisplaystyle % "toc display" as if etoc was not loaded
\etocstandardlines % "toc lines as if etoc was not loaded

\tableofcontents % Output the table of contents

\listoffigures % Output the list of figures

% Comment both of the following lines to have the LOF and the LOT on different pages
\let\cleardoublepage\bigskip
\let\clearpage\bigskip

\listoftables % Output the list of tables

\endgroup

%----------------------------------------------------------------------------------------
%	MAIN BODY
%----------------------------------------------------------------------------------------

\mainmatter % Denotes the start of the main document content, resets page numbering and uses arabic numbers
\setchapterstyle{kaoCJK} % Choose the default chapter heading style

\chapter{第一章}

嗨 kaobookCJK 在此。\sidecite{James2013}

\blindtext

\pagelayout{wide} % No margins
\addpart{基础篇}
\pagelayout{margin} % Restore margins

\chapter{第二章}

\blindtext

\appendix % From here onwards, chapters are numbered with letters, as is the appendix convention

\pagelayout{wide} % No margins
\addpart{附录}
\pagelayout{margin} % Restore margins

\chapter{其他}

\blindtext

%----------------------------------------------------------------------------------------

\backmatter % Denotes the end of the main document content
\setchapterstyle{plain} % Output plain chapters from this point onwards

%----------------------------------------------------------------------------------------
%	BIBLIOGRAPHY
%----------------------------------------------------------------------------------------

% The bibliography needs to be compiled with biber using your LaTeX editor, or on the command line with 'biber main' from the template directory

\defbibnote{bibnote}{\noindent 以下依次列出所有参考文献.\par\bigskip} % Prepend this text to the bibliography
\printbibliography[heading=bibintoc, title=参考文献, prenote=bibnote] % Add the bibliography heading to the ToC, set the title of the bibliography and output the bibliography note

%----------------------------------------------------------------------------------------
%	INDEX
%----------------------------------------------------------------------------------------

% The index needs to be compiled on the command line with 'makeindex main' from the template directory

\printindex % Output the index

\end{document}
