\setchapterpreamble[u]{\margintoc}
\chapter{文本标注}
\labch{文本标注}

边注是 1.5 栏宽边排版书籍的显著特性。确实,宽边距意味着更多的信息可以被显示出来,常用于放置带编号的边注、不带编号的边注、较小的章节目录、参考文献以及一些特别的块。

\section{带编号边注}

边注与脚注的区别就是其位置在页边距内,这样可读性会更高。插入一个边注可以使用命令 \Command{sidenote\{说明文本\}}。你可以使用命令 \Command{sidenote[记号]\{文本\}} 指定一个记号,\sidenote[O]{这条边注有一个 O !}也可以设置一个偏置距离,使文本向上或向下移动,完整语法如下:

\begin{lstlisting}[style=kaolstplain]
\sidenote[记号][偏置距离]{文本}
\end{lstlisting}

如果设置偏置距离,则记号的括号必须保留,内容可以为空。\sidenote{了解更多关于命令 \Command{sidenote} 的用法,请阅读 \Package{sidenotes} 包的文档。}

在 \Class{kaobookCJKsc} 模板中,我们从 \Package{snotez} 包中拷贝了一个特性:利用 \Command{baselineskip} 命令实现以行为单位的偏移。比如下面这样的语法会向上移动一行:

\begin{lstlisting}[style=kaolstplain]
\sidenote[][*-1]{边注文本}
\end{lstlisting}

刚才说过,边注由 \Package{sidenotes} 包提供,但此包又依赖 \Package{marginnote} 包。

\section{无编号边注}

这个命令,与前一节介绍的命令非常相似。可以用 \Command{marginnote[偏置]\{文本\}} 命令创建无编号的边注,同样包含偏置距离参数,以及基于 \Command{baselineskip} 命令实现的偏置行数参数,等等。\marginnote[-3cm]{虽然无编号边注命令来自 \Package{marginnote} 包,但是已经重新定义以便调整设置偏置参数的位置,目前该参数在边注文本之前设定,而原版命令在文本之后。同样,也基于 \Command{baselineskip} 增加了按行偏置的功能。这些设计目的是让所有命令有个统一的形式,防止太多命令需要记忆!}

\begin{lstlisting}[style=kaolstplain]
\marginnote[-12pt]{文本} or \marginnote[*-3]{文本}
\end{lstlisting}

\begin{kaobox}[frametitle=To Do]
A small thing that needs to be done is to renew the \Command{sidenote}
command so that it takes only one optional argument, the offset. The
special mark argument can go somewhere else. In other words, we want the
syntax of \Command{sidenote} to resemble that of \Command{marginnote}.
\end{kaobox}

我们加载了 \Package{marginnote}、\Package{marginfix} 和 \Package{placeins} 三个包。既然 \Package{sidenotes} 继承了 \Package{marginnote},我们之前讲的关于有编号边注同样适用于此。两种边注都做了轻微的调整(\texttt{\textbackslash renewcommand\{\textbackslash marginnote\linebreak vadjust\}\{3pt\}}),以对齐底部。重要的一点是,两种边注不指定偏置距离时,都采用垂直浮动布局。但是,一旦指定了偏置距离,浮动会禁用。少数情况下会出现一些问题,有时可以设置偏置距离为 0pt 以便消除浮动。

\section{脚注}

我们在这里讨论脚注,虽然它不显示在边注中,而且其作用也被边注替代。脚注强迫读者转移到页面的另一个区域。有证据表面,边注解决了这个问题。但无论如何,作为竞争者,我们保留了这个命令 \Command{footnote},以便你可以使用。\footnote{脚注以这样的形式存在,PDF 文件也会生成一个引用。}

\section{章节边注目录}
\labsec{martoc}

既然我们谈到了边注,这里再引入一个命令 \Command{margintoc} 用于在边注内生成一个小目录。和我们讨论过的其他命令一样,\Command{margintoc} 接受一个偏置距离参数,像这样:\Command{margintoc[offset]}。

命令可以用于文档任何位置,但我们认为在章节开头比较有意义。在这份文档里,我使用了 \KOMAScript 中的一个特性,并在导言区加入了这样的配置:

\marginnote{边注目录的字体和文档目录字体相同。}

\begin{lstlisting}[style=kaolstplain]
\setchapterpreamble[u]{\margintoc}
\chapter{Chapter title}
\end{lstlisting}

边注的空间是一个重要的资源,所以有可能需要在边注打印简短版的章节目录。于是节(小节)标题可能有三种:正文标题、文档目录标题、边注目录标题。这些不同版本可以使用如下命令同时设定:

\begin{lstlisting}[style=kaolstplain]
\section[alternative-title-for-toc]{title-as-written-in-text}[alternative-title-for-margintoc]
\end{lstlisting}

默认情况下,边注目录只包含节标题和小节标题,如果你只想展示节标题,添加如下命令到导言区:

\begin{lstlisting}[style=kaolstplain]
\setcounter{margintocdepth}{\sectiontocdepth}
\end{lstlisting}

\section{旁白列表}

On some occasions it may happen that you have a very short piece of code
that doesn't look good in the body of the text because it breaks the
flow of narration: for that occasions, you can use a
\Environment{marginlisting}. The support for this feature is still
limited, especially for the captions, but you can try the following
code:

\begin{marginlisting}[-1.35cm]
	\caption{An example of a margin listing.}
	\vspace{0.6cm}
	\begin{lstlisting}[language=Python,style=kaolstplain]
print("Hello World!")
	\end{lstlisting}
\end{marginlisting}

\begin{verbatim}
\begin{marginlisting}[-0.5cm]
	\caption{My caption}
	\vspace{0.2cm}
	\begin{lstlisting}[language=Python,style=kaolstplain]
	... code ...
	\end{lstlisting}
\end{marginlisting}
\end{verbatim}

Unfortunately, the space between the caption and the listing must be
adjusted manually; if you find a better way, please let me know.

Not only textual stuff can be displayed in the margin, but also figures.
Those will be the focus of the next chapter.
