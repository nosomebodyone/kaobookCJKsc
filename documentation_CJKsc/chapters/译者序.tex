\setchapterimage{seaside}
\chapter*{译者序} % 加星号防止多余书签
\addcontentsline{toc}{chapter}{译者序} % 添加到目录

无意间发现这份模板,感觉非常简约大方,所以就想尝试一下。不巧的是,作者没有考虑多语言问题,简体中文并没有默认支持。浅研究一下,就有了这个版本。目前基本满足日常所需,折服于 \LaTeX 的强大,是时候替换 Markdown 了。

下面简单说明一下 kaobookCJKsc 模板。首先,原则上\textbf{尽量减少对原模板的修改,以便于未来更新。}但是有些设置,一方面由于作者硬编码在样式文件中,另一方面由于很多包不支持中文,所以并不会自动翻译。为了处理这样的冲突,不得不拷贝一份,重新移植。主要工作有以下几个方面:

\begin{itemize}
    \item 为了区分原模板,所有文件增加了 CJKsc 后缀。
    \item 增加了类似 \texttt{kaoWinCJKsc.sty} 这样的配置包,尽可能将中文相关的配置以及平台依赖集中到这里。一来为跨平台提供便利,二来也能避免配置的杂乱无章。
    \item 为了项目结构的整洁,加入了 styles 子文件夹。
    \item 为了支持多平台,默认依赖 Google Noto 和 liberation 系列字体。
    \item 一些中文语序和英文相反,由于配置无法放在导言区或包内,单独增加一个 \texttt{config.tex} 文件保存,在正文中导入即可。
\end{itemize}

至此,一份漂亮的图书模板就做好了。本着够用就好的原则,没必要浪费太多时间在工具配置上。那么,万事具备,只欠东风,剩下的事情就交给你了。用好这份模板,多多创作吧!

\begin{flushright}
	\textit{薛浩}\hspace*{2.5em}

    \textit{\zhtoday}
\end{flushright}
