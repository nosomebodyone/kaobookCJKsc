% Load the kaohandt class (with the default options)
\documentclass[
	%fontsize=10pt, % Base font size
	%twoside=false, % If true, use different layouts for even and odd pages (in particular, if twoside=true, the margin column will be always on the outside)
	%secnumdepth=2, % How deep to number headings. Defaults to 2 (subsections)
	%abstract=true, % Uncomment to print the title of the abstract
]{styles/kaohandtCJKsc}


%----------------------------------------------------------------------------------------
%	包以及文档配置,新项目可以直接复制粘贴
%----------------------------------------------------------------------------------------

% 加载排版包(不支持 chinese)
\usepackage{polyglossia}
\setmainlanguage{english}
\SetLanguageKeys{english}{indentfirst=true}
% 加载引号风格包(不支持 chinese)
\usepackage[style=english]{csquotes}
% 加载参考文献相关设置
\usepackage[style=gb7714-2015ms,backref=false]{styles/kaobiblioCJKsc} % 默认使用 GB 标准
\addbibresource{main.bib} % 参考文献文本
% 加载数学相关设置
\usepackage[framed]{styles/kaotheoremsCJKsc}
% 加载超链接相关设置
\usepackage{styles/kaorefsCJKsc}
% 加载平台中文字体及相关设置
\usepackage[testMathBox]{styles/kaoWinCJKsc}
% 加载测试包
\usepackage{blindtext}

% 图片路径
\graphicspath{{examples/documentation/images/}{images/}}
% 生成用于编译 索引 的文件
\makeindex[columns=3, title=索引, intoc]
% 生成用于编译 术语 的文件
\makenomenclature
% 允许标题中出现 \\ 而不发出警告
\pdfstringdefDisableCommands{\let\\\empty}

%----------------------------------------------------------------------------------------
\begin{document}

%----------------------------------------------------------------------------------------
%	加载其他无法放在导言区的配置
%----------------------------------------------------------------------------------------
%%%%%%%%%%%%%%%无法放在导言区的设置%%%%%%%%%%%%%%%%%%%%
%目录中文名
\renewcommand{\contentsname}{目录}
\renewcommand{\listfigurename}{插图}
\renewcommand{\listtablename}{表格}
\renewcommand{\lstlistlistingname}{清单}
\renewcommand{\lstlistingname}{清单}
\renewcommand{\indexname}{索引}
\renewcommand{\bibname}{参考文献}
%其他中文名
\renewcommand{\pagename}{页}
\renewcommand{\chaptername}{章}
\renewcommand{\sectionname}{节}
\renewcommand{\subsectionname}{小节}
\renewcommand{\figurename}{图}
\renewcommand{\tablename}{表}
\renewcommand{\eqname}{公式}
\renewcommand{\defname}{定义}
\renewcommand{\assumname}{假设}
\renewcommand{\thmname}{定理}
\renewcommand{\propname}{命题}
\renewcommand{\lemmaname}{引理}
\renewcommand{\remarkname}{备注}
\renewcommand{\examplename}{例}
\renewcommand{\exercisename}{练习}
%引用语序
\renewcommand{\refpage}[1]{\hyperref[#1]{\textcolor{Blue}{第\xspace\pageref{page:#1}\xspace\pagename}}} % 第 84 页
\renewcommand{\refch}[1]{\hyperref[ch:#1]{\textcolor{Blue}{第\xspace\ref{ch:#1}\xspace\chaptername}}} % 第 7 章
\renewcommand{\nrefch}[1]{\hyperref[ch:#1]{\textcolor{Blue}{第\xspace\ref{ch:#1}\xspace\chaptername(\nameref{ch:#1})}}} % 第 7 章(数学)
\renewcommand{\refsec}[1]{\hyperref[sec:#1]{\textcolor{Blue}{第\xspace\ref{sec:#1}\xspace\sectionname}}}
\renewcommand{\nrefsec}[1]{\hyperref[sec:#1]{\textcolor{Blue}{第\xspace\ref{sec:#1}\xspace\sectionname(\nameref{sec:#1})}}}
\renewcommand{\refsubsec}[1]{\hyperref[subsec:#1]{\textcolor{Blue}{第\xspace\ref{subsec:#1}\xspace\subsectionname}}}
\renewcommand{\nrefsubsec}[1]{\hyperref[subsec:#1]{\textcolor{Blue}{第\xspace\ref{subsec:#1}\xspace\subsectionname(\nameref{subsec:#1})}}}
%脚注编号
% \renewcommand{\thefootnote}{\arabic{footnote}} % 1, 2, 3
% \renewcommand{\thefootnote}{\alph{footnote}} % a, b, c
% \renewcommand{\thefootnote}{\roman{footnote}} % i, ii, iii



%----------------------------------------------------------------------------------------
%	REPORT INFORMATION
%----------------------------------------------------------------------------------------

\title[用于报告(或手册)的 KaoCJK 模板]{用于报告(或手册)的 KaoCJK 模板}

\author[MX, MF, JMC]{薛浩\thanks{Michael Faraday who created Kao} \and Michael Faraday\thanks{Royal Society of London} \and John McClane \thanks{New York City Police Department}}

\date{\zhtoday}

%----------------------------------------------------------------------------------------
%	TITLE AND ABSTRACT
%----------------------------------------------------------------------------------------

\maketitle

\margintoc

\begin{abstract}
\noindent
摘要部分,以下是测试文字。\blindtext
\end{abstract}

{\noindent\textbf{关键字:} \LaTeX,KaoCJKsc,手册,文章,报告}

\medskip

%----------------------------------------------------------------------------------------
%	MAIN BODY
%----------------------------------------------------------------------------------------

\section{介绍}

此处写下你的介绍性文字,\sidecite{James2013} 并确保引用你的源文件。

\blindtext\sidenote{这里是带编号边注}

\section{方法}

\appendix % From here onwards, chapters are numbered with letters, as is the appendix convention

\section{附录}

\blindtext

%----------------------------------------------------------------------------------------
%	BIBLIOGRAPHY
%----------------------------------------------------------------------------------------

% The bibliography needs to be compiled with biber using your LaTeX editor, or on the command line with 'biber main' from the template directory

\printbibliography[title=参考文献] % Set the title of the bibliography and print the references

\end{document}
